%------------------------------------------------------------------------------
\documentclass[master,final]{cimt}
% オプションについては,マニュアルを参照.
% \documentclass[master,oneside]{cimt} 

% 必要とするパッケージがあれば,ここで指定する.
\usepackage[dvipdfmx]{graphicx}
\usepackage{url}
\usepackage{layout}

% 論文タイトル
\jtitle{これこれについて}
% 長い時には自動的に改行されるが,次のように明示的に改行することもできる.
% \jtitle{長いタイトルを\\このように改行位置を指定して組む}

% 英文タイトル
\etitle{On blahblah}

% キーワード
% \keywords{creative, informatics}

% 著者名
\jauthor{秋葉 創太}

% 英文著者名
\eauthor{Sota Akiba}

% 指導教員
\supervisor{本郷 情一 教授}

% 提出月.この例だと,2010年1月.
\handin{2010}{1}

\input preamble.tex

\begin{document}

% 表紙と表紙裏
\maketitle 

% ここから前文
\frontmatter

% 概要
\begin{jabstract}
\input src/jabst.tex
\end{jabstract}

% 英文概要
\begin{eabstract}
\input src/eabst.tex
\end{eabstract}

% 目次
\tableofcontents

% ここから本文
\mainmatter

\input src/intro.tex
\input src/body.tex
\input src/concl.tex

% ここから後付
\backmatter

% 発表文献
\pubUseLongName % 指定すると,タイトルが 「発表文献と研究活動」になる.
\begin{publications}
\input src/publications.tex
\end{publications}

% 参考文献: BibTeX を使う場合の例 (styleは適宜選択)
\bibliographystyle{junsrt}
\bibliography{main}

% 参考文献: 直接記述する場合の例
% \input biblio.tex

% 謝辞 (前文においても良い)
\begin{acknowledgements}
\input src/ack.tex
\end{acknowledgements}

%付録 (必要な場合のみ)
\appendix

\input src/appendix.tex

\end{document}
